%\documentclass[12pt,letterpaper]{article}\usepackage{epsfig,amsmath,amssymb,natbib,ifthen}\provideboolean{StandAlone}\setboolean{StandAlone}{true} \begin{document}

\subsection{Euler Equation Methods}

The early literature relevant to identifying the strength
of precautionary motives tended to rely on Euler
equation estimation (see \cite{browning&lusardi:jel} for a survey),
often by estimating regression equations of the form
\begin{eqnarray}
  \Delta \log {c}_{t+1} & = & \alpha_{0}+\alpha_{1} \mathbf{E}_{t}[r_{t+1}]
\end{eqnarray}
and interpreting the coefficient on the interest rate term as an estimate
of the inverse of the coefficient of relative risk aversion (which holds true
under time-separable CRRA utility, cf. \eqref{eq:CGrowPF}).
However, this analysis did not take into account the dependence of higher order
terms like $\phi$ on the independent variables (see \eqref{eq:phi}).  Some papers like~\cite{dynan:precautionary}
attempted to account for precautionary contributions to consumption growth;
but see \cite{carroll:death} for a critique of the whole Euler equation literature
(including the second-order approach).

\subsection{Structural Estimation Using Micro Data}

A new methodology for estimating the importance of precautionary motives was pioneered by
\cite{gpCoverLC} and \cite{cagettiWprofiles} (with a related earlier contribution
by~\cite{palumbo:medical}).  Their idea was to calibrate an explicit life cycle optimization problem using
empirical data on the magnitude of household-level income shocks, and to search econometrically for the values
of parameters such as the coefficient of relative risk aversion that maximized the model's ability to fit some
measured feature of the empirical data.  \cite{gpCoverLC} matched the profile of mean
consumption over the lifetime; \cite{cagettiWprofiles} matched the profile of median wealth.  The intensity of
the precautionary motive emerges, in each case, as an estimate of the coefficient of relative risk aversion,
which \cite{gpCoverLC} put at about 1.4 and \cite{cagettiWprofiles} finds to be somewhat
larger. (A value of 1 corresponds to logarithmic utility).  One important caution about these quantitative
results is that the method's estimates of relative risk aversion depend on the model's assumption about the
degree of risk households face.  
\begin{comment} % Original text
Recent work by \cite{lmp:wagerisk} suggests that the estimates of the magnitude
of permanent shocks in \cite{carroll&samwick:nature} used for calibration by \cite{gpCoverLC}
and \cite{cagettiWprofiles} may be overstated by as much as 50 percent.
\end{comment}
% Original revised on 2006-02-15 per email from Luigi on 2006-01-30
Recent work by \cite{lmp:wagerisk} that attempts to correct for measurement problems caused by 
job mobility suggests that the estimates of the magnitude
of permanent shocks in \cite{carroll&samwick:nature} used for calibration by \cite{gpCoverLC}
and \cite{cagettiWprofiles} may be overstated by as much as 50 percent.
Reestimation of the structural parameters
using the Low et.\ al.\ calibration would generate larger estimates of relative risk aversion.

\subsection{Regression Evidence}

A separate literature attempts direct empirical measurement of the
relationship between uncertainty and wealth.  To fix notation, index
individual households by $i$ and assume uncertainty for household $i$
in period $t$ can be measured by some variable $\sigma_{t,i}$. Then in
its simplest form the idea is to perform a regression of cash-on-hand on 
its determinants along the lines of
\begin{eqnarray}
  \log {m}_{t,i} & = & \sigma_{t,i} \gamma + Z_{t,i} \alpha + \epsilon_{t,i}
\end{eqnarray}
where $Z$ is some set of variables that capture life cycle, time
series, and other nonprecautionary effects.  In principle, one can
then calculate the predicted magnitude of ${m}$ if everyone's
uncertainty were set to zero (or some alternative like the minimum
measured value of $\sigma$ in the population).

In principle this method permits the data to speak in a much less
filtered way than the structural estimation approach.  A drawback is
that even if the magnitude of precautionary wealth could be estimated
reliably and precisely, it would not be clear how to translate those
estimates into a measure of relative risk aversion or some other set
of behavioral parameters that could be used for analyzing policy
questions such as the optimal design of unemployment insurance or
taxation.

A further disadvantage is that the method does not reliably yield the
same answer in different data.  Using a measure of subjective earnings uncertainty from a
survey of Italian households, \cite{gjt:smallPS} estimate the
precautionary component of wealth at only a few percent, while
\cite{kazarosian:restat} and \cite{carroll&samwick:howbig} estimate
the precautionary component of wealth for typical U.S.\ households to
be in the range of 20-50 percent.  \cite{hlkt:precautionary} argue
that estimates of $\alpha$ are inordinately sensitive to whether
business owners are included in the dataset; and work by
\cite{lusardi:importance,lusardi:subjective} and \cite{EngenGruber:UI}
implies much smaller precautionary wealth.  Such large variation in
empirical estimates is not plausibly attributable to actual behavioral
differences across the various sample populations.

A problem that plagues all these efforts is identifying exogenous
variations in uncertainty across households.  The standard method has
been to use patterns of variation across age, occupation, education,
industry, and other characteristics.  This runs the danger that people
who are more risk tolerant may both choose to work in a risky industry
and choose not to save much, biasing downward the estimate of the
effect of an exogenous change in risk.

One recent paper attempts to get around this problem by using a
natural experiment: \cite{fs:germany} show that before the collapse of
the Berlin Wall, East German civil servants had similar income
uncertainty to that faced by other East Germans.  However, after the
collapse of Communism, income uncertainty went up dramatically for
most East Germans - but not for civil servants, who were given
essentially the same risk-free jobs in the new merged government that
they had had before the collapse.  \cite{fs:germany} show that, in
accord with a model that includes substantial precautionary effects,
saving rates of most East Germans increased sharply after unification,
but saving rates of civil servants did not.  By contrast, the West
Germans--who would have been subject to more selection into jobs based
on risk preferences--exhibited little difference in saving rates
between civil servants and others with riskier jobs, either before or
after reunification.

\subsection{Survey Evidence}

Given the difficulties of obtaining reliable quantitative measures 
of precautionary motives using the revealed preference econometric
techniques sketched above, some researchers have turned to
approaches that involve asking survey participants more direct questions.

\cite{KennickellLusardiDisentangling} find that when respondents for
the 1995 and 1998 U.S. {\it Survey of Cosnumer Finances} are asked
their target level of precautionary wealth, most have little
difficulty answering the question; desired precautionary wealth
represents about 8 percent of total net worth and 20 percent of total
financial wealth.  They find that respondents cite a broad array of
risks in making their precautionary targets: In addition to labor
income risk, they face health risk, business risk, and the risks of
unavoidable expenditures (e.g. home repairs).  (Consumers are clearly
aware of the theoretical point that a given dollar of wealth can
provide self-insurance against multiple different kinds of risks,
since the risks are not likely to be perfectly correlated with each
other).

Carefully designed survey questions can in principle also be used to
elicit information on the strength of underlying preferences (like
risk aversion) that determine precautionary behavior.  The principle
that whenever risk-bearing increases with assets, the precautionary
saving motive (prudence) must be stronger than risk aversion provides
an important theoretical lower bound on the degree of prudence.
Using survey responses to hypothetical gambles over lifetime income in the
{\em Health and Retirement Study}, \cite{kssImputing} estimate that
relative risk aversion has a median of 6.3 and a mean of 8.2.  (Note
that because of Jensen's inequality, the mean of relative risk
aversion $E \rho$ is larger than the reciprocal of the mean of
relative risk tolerance $\frac{1}{E(1/\rho)}$.)  These estimates of
relative risk aversion imply precautionary saving motives much
stronger than those that have been used empirically to match observed
wealth holdings.  This discrepancy remains
unresolved.




\ifthenelse{\boolean{StandAlone}}{\end{document}}{}
