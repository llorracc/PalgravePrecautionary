%\documentclass[12pt,letterpaper]{article}\usepackage{epsfig,amsmath,amssymb,natbib,ifthen,hyperref} \provideboolean{StandAlone}\setboolean{StandAlone}{true}\begin{document}

Precautionary saving is additional saving that results from the
knowledge that 
% original future income is risky.
the future is uncertain.  % Want to allow for medical shocks, mortality risk, etc as part of the picture

In principle, additional saving can be achieved either by consuming
less or by working more; here, we follow most of the literature in
neglecting the ``working more'' channel by treating non-capital income as
exogenous.

Before proceeding, a terminological clarification is in order.
``Precautionary saving'' and ``precautionary savings'' are often
(understandably) confused.  ``Precautionary saving'' is a response of
current spending to future risk, conditional on current circumstances.
``Precautionary savings'' is the additional wealth owned at a given point in
time as the result of past precautionary behavior.  That is,
precautionary savings at any date is the stock of extra wealth that
results from the past flow of precautionary saving.  To avoid
confusion, we advocate use of the phrase ``precautionary wealth'' in
place of ``precautionary savings.''

\ifthenelse{\boolean{StandAlone}}{\end{document}}{}
